\section{Data testing}
The package \Biocpkg{blima} provides a basic infrastructure for performing bead level and probe level testing of the data by means of functions \Rfunction{doTTests} and \Rfunction{doProbeTTests}.

Here we show how to proceed if we want to know 10 most differentially expressed probes (and associate genes) between groups A and E. We know that in the object \Robject{blimatesting} from the \Rpackage{blimaTestingData} package there is one extra array. We have no intention to include this array into the analysis. By means of the parameter \Robject{normalizationMod} of the functions \Rfunction{bacgroundCorrect}, \Rfunction{nonPositiveCorrect}, \Rfunction{varianceBeadStabilise} and \Rfunction{quantileNormalize} we can choose a subset of the arrays in our list of \Rclass{beadLevelData} objects for the further processing. The \Robject{normalizationMod} specifies a list of logical vectors with the same structure as the first input object of given function, typically list of \Rclass{beadLevelData} objects or single \Rclass{beadLevelData} object. In the \Robject{normalizationMod} object the logical value TRUE at certain position means to process corresponding chip spot, logical value FALSE means to exclude corresponding chip slot from processing.

First we make logical lists corresponding to arrays to process and groups of arrays. 