\section{Summarization}
Even though the \Biocpkg{blima} provides methods to work with unsummarized data it may be necesery sometimes to work also with summarized data. In the particular application mentionned in this manual we export the data to the format accepted by Gene Expression Omnibus. We need to have data summarized according to the Illumina Probe ID. This summarization the data from any slot is implemented in the function \Rfunction{createSummarizedMatrix}.

We allready prepared \Robject{adrToIllumina} mapping thus we create summarized matrix by calling the \Rfunction{createSummarizedMatrix}. Then we translate the ArrayAddressID to the ProbeID and create the output matrces in the form acceptable by the Gene Expression Omnibus. We prepare two matrices, first based on the data from the "qua" slot and second based on the "GrnF" slot. The "qua" matrix is called normalized data and "GrnF" matrix is called non normalized data.

The data for dataset dataset \href{http://www.ncbi.nlm.nih.gov/geo/query/acc.cgi?acc=GSE56129}{GSE56129} was prepared in the similar fashion using \Robject{annotationHumanHT12V4} object.